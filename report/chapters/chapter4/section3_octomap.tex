\section{Octomap}
\label{section:octomap}

Η βιβλιοθήκη OctoMap \cite{hornung13auro} υλοποιεί μία αναπαράσταση του τρισδιάσταστου χώρου, παρέχοντας τις απαραίτητες δομές δεδομένων και αλγορίθμους χαρτογράφησης σε C++, με τρόπο κατάλληλο για χρήση σε  ρομποτικές εφαρμογές. Η υλοποίηση του χάρτη είναι σε μορφή οκταδικού δέντρου (octree), το οποίο προσφέρει σημαντικά πλεονεκτήματα. Είναι δυνατή η απεικόνιση τόσο του ελεύθερου όσο και του άγνωστου χώρου, ενώ παράλληλα υποστηρίζεται η προσθήκη νέων πληροφοριών οποιαδήποτε στιγμή στο χάρτη. Στο \autoref{fig:octree} φαίνεται η αποθήκευση ελεύθερων και κατειλημμένων κελιών στο οκταδικό δέντρο. Η ανανέωση γίνεται με πιθανοτικό τρόπο, καθώς λαμβάνει υπόψη πιθανό θόρυβο των αισθητήρων και δυναμικές μεταβολές του περιβάλλοντος. Επίσης, το μέγεθος του χάρτη δεν χρειάζεται να είναι γνωστό εκ των προτέρων. Είναι δυνατή η δυναμική επέκταση του και η προβολή του με διαφορετική ανάλυση, ανάλογα με τις ανάγκες του προβλήματος. Τέλος, η αποθήκευση του σε μορφή octree, οδηγεί στην αποδοτική αποθήκευση του τόσο στη μνήμη, όσο και στο δίσκο.

\begin{figure}[!ht]
    \centering
    \includegraphics[width=0.8\textwidth]{./images/chapter4/octree.png}
    \caption{Ογκομετρική απεικόνιση σε octree και το αντίστοιχο οκταδικό δέντρο} 
    \label{fig:octree}
\end{figure}

