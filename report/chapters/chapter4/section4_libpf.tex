\section{Βιβλιοθήκη Particle Filter}
\label{section:libpf}

Η βιβλιοθήκη που υλοποιεί το φίλτρο σωματιδίων για τον αλγόριθμο εκτίμησης θέσης είναι η libPF\footnote{\href{https://github.com/stwirth/libPF}{https://github.com/stwirth/libPF}}. Το φίλτρο σωματιδίων είναι μια συλλογή καταστάσεων (States) τις οποίες παρακολουθεί, τις αξιολογεί με βάση μία τιμή βάρους που προκύπτει από το μοντέλο αξιολόγησης και θεωρεί την κατάσταση με το μεγαλύτερο βάρος ως την πραγματική. Έπειτα, πραγματοποιείται μια δειγματοληψία των σωματιδίων, ανάλογα με την κατάσταση των βαρών και αξιολογούνται τα επόμενα σωματίδια.

Για την λειτουργία του φίλτρου απαιτείται ο ορισμός την κατάστασης που παρακολουθείται, ενός μοντέλου κίνησης και ενός μοντέλου αξιολόγησης για τις καταστάσεις. Εκτός αυτών, μπορεί να οριστεί και ένα μοντέλο κατανομής καταστάσεων, για την περίπτωση που η αρχική θέση είναι γνωστή και παρέχεται στο φίλτρο.

Όσον αφορά τη διαδικασία που ακολουθεί ένα Particle Filter, αρχικά δημιουργείται το φίλτρο και προσδιορίζονται οι παράμετροι του μη γραμμικού συστήματος. Εάν θέλουμε να λύσουμε το πρόβλημα της καθολικής εύρεσης θέσης (Global Localization) τα σωματίδια αρχικοποιούνται σε τυχαίες θέσεις μέσα στο χώρο, ενώ αν γνωρίζουμε την αρχική θέση τα σωματίδια αρχικοποιούνται με βάση μια κανονική κατανομή γύρω από αυτήν. Το αρχικό βάρος όλων των σωματιδίων είναι ίδιο. Στην συνέχεια, τα σωματίδια κινούνται σύμφωνα με το κινηματικό μοντέλο του ρομποτικού πράκτορα, όπως αυτό λαμβάνεται από την οδομετρία ή άλλους σχετικούς αισθητήρες. Επομένως, τα σωματίδια βρίσκονται σε μια καινούρια κατάσταση και πρέπει να αξιολογηθούν, ώστε να εκτιμηθεί η νέα τιμή του βάρους τους. Η αξιολόγηση γίνεται σύμφωνα με τις μετρήσεις του αισθητήρα απόστασης που βρίσκεται στο drone. Η επαναδειγματοληψία των σωματιδίων γίνεται όταν ο αριθμός των αποτελεσματικών σωματιδίων ($N_{eff}$) είναι μικρότερος από το μισό του συνολικού αριθμού των σωματιδίων. Η τιμή του $N_{eff}$ δίνεται από τον παρακάτω τύπο:
\begin{equation*}
     N_{eff} = \frac{1}{\sum_{i=1}^{N_s} (w_k^i)^2}
\end{equation*}

Η βιβλιοθήκη libPF επιτρέπει επίσης την εκτέλεση της επαναδειγματοληψίας είτε σε κάθε βήμα εκτέλεσης του φίλτρου, μια δυνατότητα που επιβαρύνει υπολογιστικά το σύστημα, ή να μην συμβαίνει ποτέ, με αποτέλεσμα να υπάρχει η πιθανότητα κάποια σωματίδια να μηδενίσουν το βάρος τους μετά από κάποια βήματα εκτέλεσης του φίλτρου και να χαθεί η επίδρασή τους.