\section{Πλήρης Κάλυψη Χώρου}
\label{section:coverage}

Η πλήρης κάλυψη ενός χώρου πρόκειται για το πρόβλημα της δημιουργίας ενός μονοπατιού το οποίο διέρχεται από όλα τα σημεία ενδιαφέροντος ενός περιβάλλοντος, καθώς παράλληλα γίνεται αποφυγή εμποδίων. Σύμφωνα με το \cite{galceran2013}, οι έξι απαιτήσεις της διαδικασίας πλήρους κάλυψης χώρου είναι οι εξής:
\begin{itemize}

    \item{Το ρομπότ πρέπει να διασχίσει όλα τα σημεία στην περιοχή ενδιαφέροντος, καλύπτοντας την πλήρως}
    \item{Το ρομπότ πρέπει να καλύπτει την περιοχή, χωρίς την ύπαρξη αλληλοεπικαλυπτόμενων διαδρομών}
    \item{Απαιτούνται συνεχείς και διαδοχικές διεργασίες, χωρίς την επανάληψη καμίας τροχιάς}
    \item{Το ρομπότ πρέπει να αποφεύγει κάθε είδους εμπόδιο}
    \item{Απλοϊκές τροχιές (π.χ. ευθείες ή κυκλικές κινήσεις) θα έπρεπε να χρησιμοποιούνται, καθώς προσφέρουν απλότητα στην κίνηση}
    \item{Ένα βέλτιστο μονοπάτι προτιμάται, εφόσον αυτό είναι εφικτό}

\end{itemize}

Υπάρχουν δύο κατηγορίες προσεγγίσεων για το θέμα αυτό, η απόλυτη και η ευριστική. Κατά την πρώτη, ο χώρος διαχωρίζεται σε τμήματα και μπορεί να εγγυηθεί την πλήρη κάλυψη του χώρου, ενώ κατά την ευριστική μέθοδο το ρομπότ ακολουθεί ένα σύνολο απλών κανόνων που επηρεάζουν την κίνηση του και πιθανόν να μην οδηγήσουν με επιτυχία στην πλήρη κάλυψη.

Στο \cite{7496385} παρουσιάζεται μία απόλυτη προσέγγιση στο πρόβλημα της πλήρους κάλυψης χώρου για UAVs. Ο χώρος χωρίζεται σε πολυγωνικά τμήματα με μέγεθος που επηρεάζεται από το πεδίο όρασης (Field of View) της κάμερας. Το κέντρο κάθε πολυγώνου θεωρείται ως το σημείο που πρέπει να βρεθεί το drone, ώστε να καλυφθεί η περιοχή γύρω του. Στη συνέχεια, χρησιμοποιείται ένας wavefront propagation αλγόριθμος για να βρει όλα τα δυνατά μονοπάτια που μπορούν να ενώσουν τα σημεία αυτά. Για κάθε ένα μονοπάτι υπολογίζεται το συνολικό του κόστος και επιλέγεται αυτό με το μικρότερο δυνατό. Στο τέλος, η πορεία ομαλοποιείται χρησιμοποιώντας πολυωνυμικές συναρτήσεις. Ένα σημαντικό χαρακτηριστικό της προσέγγισης αυτής είναι ότι κατά την δημιουργία των μονοπατιών, λαμβάνεται υπόψη η περιστροφική κίνηση που απαιτείται από το drone για να μεταβεί στο επόμενο σημείο, καθώς αυτή αυξάνει το χρόνο εκτέλεσης και είναι επιθυμητή η ελαχιστοποίηση των στροφών κατά την κίνηση στο χώρο.

Οι Bircher κ.α. \cite{bircher2015} παρουσιάζουν έναν αλγόριθμο για την τρισδιάστατη κάλυψη χώρου που στοχεύει στην αυτόνομη διαδικασία της επιθεώρησης χρησιμοποιώντας μη επανδρωμένα αεροσκάφη. Στην προσέγγιση αυτή, για κάθε σημείο του χάρτη υπολογίζουμε τη θέση η οποία επιφέρει την καλύτερη δυνατή θέαση του σημείου αυτού, με βάση τα χαρακτηριστικά των αισθητήρων που διαθέτει το ρομπότ. Στην συνέχεια, εφαρμόζεται μια επαναληπτική διαδικασία πεπερασμένων βημάτων, μέσω της οποίας συνδέονται τα σημεία που ελαχιστοποιούν το κόστος, είτε αυτό είναι η απόσταση είτε ο χρόνος εκτέλεσης. Η καλύτερη δυνατή σύνδεση των σημείων επιτυγχάνεται μέσω της επίλυσης του προβλήματος του περιοδεύοντος εμπόρου (Travelling Salesman Problem).

Μια αρκετά διαφορετική προσέγγιση παρουσιάζεται στο \cite{1262545}. Πιο συγκεκριμένα, χρησιμοποιούνται νευρωνικά δίκτυα για την πλήρη κάλυψη στατικού και μεταβαλλόμενου χώρου σε πραγματικό χρόνο. Η υλοποίηση αυτή εφαρμόζεται για χάρτη δύο διαστάσεων, μπορεί να καλύψει την ύπαρξη ενός ή περισσοτέρων ρομποτικών πρακτόρων και παρουσιάζει αξιόπιστα αποτελέσματα. Ο χώρος διακριτοποιείται με βάση το μέγεθος του ρομπότ και τα σημεία που προκύπτουν ενώνονται μέσω των νευρώνων του δικτύου. Η δυναμική μορφή του δικτύου μπορεί να μεταβάλλει στιγμιαία τα μονοπάτια που δημιουργούνται με βάση την προσθήκη ή την αφαίρεση εμποδίων.

Οι ευριστικές προσεγγίσεις βασίζονται στην κάλυψη του χώρου με διάφορα μοτίβα, όπως για παράδειγμα υλοποιείται στο \cite{DiFranco2016}. Στην περίπτωση αυτή, παρουσιάζεται ένας αλγόριθμος ο οποίος δημιουργεί ένα μονοπάτι που ικανοποιεί, εκτός των άλλων, και τους ενεργειακούς περιορισμούς του ρομπότ. Μετά την δημιουργία μιας πορείας που αποτελείται κυρίως από κινήσεις προς τα μπροστά και πίσω, εξετάζεται η ταχύτητα με την οποία πρέπει να κινείται το drone, ώστε να ελαχιστοποιείται η κατανάλωση ενέργειας.

