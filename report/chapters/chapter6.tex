\chapter{Πειράματα - Αποτελέσματα}
\label{chapter:experiments}

Τα πειράματα χωρίζονται σε δύο φάσεις. Η πρώτη αφορά το σύστημα εντοπισμού θέσης και η δεύτερη το σύστημα πλήρους κάλυψης χώρου από το drone. Για την εκτέλεση των πειραμάτων χρησιμοποιήθηκε υπολογιστής με τα παρακάτω χαρακτηριστικά:
\begin{table}[H]
    \begin{center}
        \small
        \caption{Χαρακτηριστικά συστήματος που χρησιμοποιήθηκε για την εκτέλεση των πειραμάτων}
        \label{tab:pc_specs}
        \begin{tabular}{| c | c | c | c |}
        \hline
        \rowcolor{Gray}
        Επεξεργαστής & Κάρτα Γραφικών & Μνήμη RAM & Λειτουργικό Σύστημα \\
        Intel® Core™ i7-7500 & Intel® HD Graphics 620 & 8 GB & Ubuntu 16.04 64bit\\
        \hline
        \end{tabular}
    \end{center}
\end{table}

Οι εκδόσεις των βιβλιοθηκών και εργαλείων που χρησιμοποιήθηκαν είναι:
\begin{table}[H]
    \begin{center}
        \caption{Εκδόσεις βιβλιοθηκών και εργαλείων που χρησιμοποιήθηκαν}
        \label{tab:tools_versions}
        \begin{tabular}{| c | c | c | c |}
        \hline
        \rowcolor{Gray}
        ROS & Gazebo & OctoMap & OMPL \\
        Kinetic & 7.0.0 & 1.8.1 & 1.2.1 \\
        \hline
        \end{tabular}
    \end{center}
\end{table}

Για την εκτέλεση των πειραμάτων, τα δεδομένα που απαιτούνται για την εξαγωγή αποτελεσμάτων αποθηκεύτηκαν σε αρχεία rosbag\footnote{\href{http://wiki.ros.org/rosbag}{http://wiki.ros.org/rosbag}}. Στη συνέχεια, έγινε εξαγωγή τους σε αρχεία μορφής \emph{.csv} και χρησιμοποιήθηκε κώδικας σε Python για τον υπολογισμό μετρικών και τη δημιουργία διαγραμμάτων.

\section{Εντοπισμός Θέσης}
\label{section:localization}

Η εύρεση της θέσης ενός ρομπότ σε εσωτερικό περιβάλλον παρουσιάζει σημαντικές δυσκολίες. Η έλλειψη του αισθητήρα GPS οδηγεί στην ανάγκη για εύρεση άλλων αξιόπιστων μεθόδων υπολογισμού της θέσης και του προσανατολισμού ενός UAV. Αρκετά συχνά συναντάμε στην βιβλιογραφία προσεγγίσεις που βασίζονται κατά κύριο λόγο σε μεθόδους υπολογιστικής όρασης. Αυτό συμβαίνει, καθώς το κόστος, το μικρό βάρος αλλά και η πληροφορία που μπορεί να παρέχει μια RGB-D κάμερα διευκολύνουν την επίλυση του προβλήματος. Το βασικό μειονέκτημα των μεθόδων αυτών είναι η πολυπλοκότητα της επεξεργασίας και οι υψηλές υπολογιστικές απαιτήσεις για επεξεργασία των δεδομένων σε πραγματικό χρόνο. Το πρόβλημα του εντοπισμού θέσης μπορεί να χωριστεί σε δύο διαφορετικές κατηγορίες, ανάλογα με το εάν είναι γνωστή η αρχική θέση του ρομπότ μέσα στο χώρο ή όχι. 

Οι Perez-Grau κ.α. \cite{francisco2017perez} προτείνουν την χρήση του κλασικού αλγορίθμου Monte Carlo Localization \cite{amcl}, ο οποίος χρησιμοποιεί πιθανοτικές μεθόδους και φίλτρο σωματιδίων για την εύρεση της θέσης, σε συνδυασμό με μία κάμερα RGB-D και ορισμένους ραδιοπομπούς τοποθετημένους σε γνωστά σημεία μέσα στο χάρτη. Η απαιτούμενη οδομετρία για τα σωματίδια του AMCL προέρχεται από οπτική οδομετρία (visual odometry), ενώ ο υπολογισμός της απόστασης από τους ραδιοπομπούς βοηθάει στην εξάλειψη του συσσωρευμένου σφάλματος που προκύπτει. Επίσης, κάθε λήψη του αισθητήρα βάθους δημιουργεί μια τρισδιάστατη αναπαράσταση του χώρου. Αυτή συγκρίνεται με τον ήδη υπάρχοντα χάρτη και εάν βρεθούν κοινά σημεία, υπολογίζεται μία εκτίμηση της θέσης του ρομπότ μέσα στο γνωστό περιβάλλον και η πληροφορία αυτή ενισχύει την εκτίμηση του AMCL.

Μία άλλη προσέγγιση είναι η χρήση πολλαπλών RGB καμερών, όπως παρουσιάζεται στο \cite{6224750}. Παρόλο που δεν αναφέρεται σε UAV, το αποτέλεσμα είναι η εκτίμηση θέσης με 6 βαθμούς ελευθερίας, που σημαίνει ότι μπορεί να χρησιμοποιηθεί και για drones. Συγκεκριμένα, χρησιμοποιείται μια προ-επεξεργασμένη αναπαράσταση του χώρου σε μορφή τρισδιάστατων σημείων (PointCloud) και μια συνάρτηση κόστους, ώστε να υπολογιστεί η θέση με βάση την απόσταση των διαδοχικών λήψεων και την απόκλιση των λήψεων από τον γνωστό χάρτη.

Οι Ok, Greene και Roy \cite{7487651} χρησιμοποιούν μία RGB-D κάμερα, μόνο για την δημιουργία του χάρτη, ενώ η διαδικασία του εντοπισμού θέσης βασίζεται αποκλειστικά σε RGB εικόνα. Συγκεκριμένα, δημιουργούν εικονικές λήψεις του περιβάλλοντος από διάφορα σημεία κατανεμημένα στο χάρτη και στη συνέχεια κάθε εικόνα που λαμβάνεται από την κάμερα συγκρίνεται με τις αρχικές λήψεις, χρησιμοποιώντας οπτικά χαρακτηριστικά. Είναι σημαντικό να αναφερθεί ότι η υλοποίηση αυτή δεν απαιτεί μεγάλη υπολογιστική ισχύ, λόγω της μείωσης του ρυθμού λήψης εικόνων.

Οι Beul κ.α. \cite{Beul2017} αναφέρονται στην αυτόνομη πλοήγηση των UAVs σε μεγάλες αποθήκες. Οι αισθητήρες που χρησιμοποιούνται είναι ένα οριζόντιο και ένα κάθετο lidar, γωνίας 270° το κάθε ένα, τρία ζευγάρια από στερεοσκοπικές κάμερες, ένα IMU και ένας αναγνώστης RFID. Με χρήση της οπτικής οδομετρίας, υπολογίζεται η κίνηση του drone, ενώ οι αποστάσεις που προκύπτουν από τους αισθητήρες laser, μετατρέπονται σε μορφή PointCloud και συγκρίνονται οι επιφάνειες του με τις επιφάνειες του υπάρχοντα χάρτη. Όσον αφορά την πλοήγηση του drone στο χώρο, χρησιμοποιείται ο αλγόριθμος A* για την εύρεση μονοπατιών σε έναν διακριτοποιημένο κόσμο σε μορφή OctoMap, ενώ παράλληλα χρησιμοποιείται και τοπική αποφυγή εμποδίων για την περίπτωση ύπαρξης δυναμικού περιβάλλοντος.

Μία ακόμη εφαρμογή με UAVs σε χώρο αποθήκης παρουσιάζεται στο \cite{8392775}. Στην περίπτωση αυτή, χρησιμοποιούνται έναν αισθητήρας απόστασης lidar γωνίας 360° και εύρους 100 μέτρων, δύο κάμερες, ένα IMU και ένας αναγνώστης RFID. Η αποτελεσματικότητα της μεθόδου φαίνεται από το γεγονός ότι το drone είχε τη δυνατότητα να κινείται με σχετικά μεγάλη ταχύτητα (2.1 μέτρα/δευτερόλεπτο) χωρίς αυτό να επηρεάζει την αξιοπιστία της εκτίμησης θέσης και της πλοήγησης. Με την δημιουργία τρισδιάστατων προσωρινών χαρτών στους οποίους προστίθενται συνεχώς τα σημεία που προκύπτουν από τον αισθητήρα απόστασης, διευκολύνεται η εύρεση της θέσης με βάση τον αρχικό χάρτη.

Σε όλες τις προηγούμενες περιπτώσεις, το drone μπορούσε να διατηρεί μία απόσταση ασφαλείας από τα αντικείμενα του χώρου. Αντίθετα, στο \cite{Fang-2017-5588} χρησιμοποιείται ένα μη επανδρωμένο αεροσκάφος για τον έλεγχο και την εκτίμηση καταστροφών μέσα σε ένα περιβάλλον πλοίου με χαμηλή ορατότητα. Το περιβάλλον αυτό περιέχει στενά περάσματα, πόρτες και μικρά αντικείμενα, συνθήκες οι οποίες δεν επιτρέπουν την χρήση των μεθόδων που παρουσιάστηκαν στις προηγούμενες βιβλιογραφικές αναφορές. Για την πλοήγηση συνδυάζεται η οπτική οδομετρία που προέρχεται από μια RGB-D κάμερα με τις μετρήσεις ενός IMU, χρησιμοποιώντας ένα φίλτρο Kalman, ώστε να υπολογιστεί η ταχύτητα. Επίσης, για την βελτίωση της εκτίμησης θέσης χρησιμοποιείται ένας αλγόριθμο που χρησιμοποιεί Φίλτρα Σωματιδίων (Particle Filters) και τροφοδοτείται από την οδομετρία. Παράλληλα, χρησιμοποιείται μια υπέρυθρη κάμερα, ώστε να μην επηρεάζεται από την χαμηλή ορατότητα του περιβάλλοντος και να ανιχνεύει τις περιοχές υψηλού κινδύνου λόγω πυρκαγιάς.

\newpage
\section{Πλήρης Κάλυψη Χώρου}
\label{section:coverage}

Η πλήρης κάλυψη ενός χώρου πρόκειται για το πρόβλημα της δημιουργίας ενός μονοπατιού το οποίο διέρχεται από όλα τα σημεία ενδιαφέροντος ενός περιβάλλοντος, καθώς παράλληλα γίνεται αποφυγή εμποδίων. Σύμφωνα με το \cite{galceran2013}, οι έξι απαιτήσεις της διαδικασίας πλήρους κάλυψης χώρου είναι οι εξής:
\begin{itemize}

    \item{Το ρομπότ πρέπει να διασχίσει όλα τα σημεία στην περιοχή ενδιαφέροντος, καλύπτοντας την πλήρως}
    \item{Το ρομπότ πρέπει να καλύπτει την περιοχή, χωρίς την ύπαρξη αλληλοεπικαλυπτόμενων διαδρομών}
    \item{Απαιτούνται συνεχείς και διαδοχικές διεργασίες, χωρίς την επανάληψη καμίας τροχιάς}
    \item{Το ρομπότ πρέπει να αποφεύγει κάθε είδους εμπόδιο}
    \item{Απλοϊκές τροχιές (π.χ. ευθείες ή κυκλικές κινήσεις) θα έπρεπε να χρησιμοποιούνται, καθώς προσφέρουν απλότητα στην κίνηση}
    \item{Ένα βέλτιστο μονοπάτι προτιμάται, εφόσον αυτό είναι εφικτό}

\end{itemize}

Υπάρχουν δύο κατηγορίες προσεγγίσεων για το θέμα αυτό, η απόλυτη και η ευριστική. Κατά την πρώτη, ο χώρος διαχωρίζεται σε τμήματα και μπορεί να εγγυηθεί την πλήρη κάλυψη του χώρου, ενώ κατά την ευριστική μέθοδο το ρομπότ ακολουθεί ένα σύνολο απλών κανόνων που επηρεάζουν την κίνηση του και πιθανόν να μην οδηγήσουν με επιτυχία στην πλήρη κάλυψη.

Στο \cite{7496385} παρουσιάζεται μία απόλυτη προσέγγιση στο πρόβλημα της πλήρους κάλυψης χώρου για UAVs. Ο χώρος χωρίζεται σε πολυγωνικά τμήματα με μέγεθος που επηρεάζεται από το πεδίο όρασης (Field of View) της κάμερας. Το κέντρο κάθε πολυγώνου θεωρείται ως το σημείο που πρέπει να βρεθεί το drone, ώστε να καλυφθεί η περιοχή γύρω του. Στη συνέχεια, χρησιμοποιείται ένας wavefront propagation αλγόριθμος για να βρει όλα τα δυνατά μονοπάτια που μπορούν να ενώσουν τα σημεία αυτά. Για κάθε ένα μονοπάτι υπολογίζεται το συνολικό του κόστος και επιλέγεται αυτό με το μικρότερο δυνατό. Στο τέλος, η πορεία ομαλοποιείται χρησιμοποιώντας πολυωνυμικές συναρτήσεις. Ένα σημαντικό χαρακτηριστικό της προσέγγισης αυτής είναι ότι κατά την δημιουργία των μονοπατιών, λαμβάνεται υπόψη η περιστροφική κίνηση που απαιτείται από το drone για να μεταβεί στο επόμενο σημείο, καθώς αυτή αυξάνει το χρόνο εκτέλεσης και είναι επιθυμητή η ελαχιστοποίηση των στροφών κατά την κίνηση στο χώρο.

Οι Bircher κ.α. \cite{bircher2015} παρουσιάζουν έναν αλγόριθμο για την τρισδιάστατη κάλυψη χώρου που στοχεύει στην αυτόνομη διαδικασία της επιθεώρησης χρησιμοποιώντας μη επανδρωμένα αεροσκάφη. Στην προσέγγιση αυτή, για κάθε σημείο του χάρτη υπολογίζουμε τη θέση η οποία επιφέρει την καλύτερη δυνατή θέαση του σημείου αυτού, με βάση τα χαρακτηριστικά των αισθητήρων που διαθέτει το ρομπότ. Στην συνέχεια, εφαρμόζεται μια επαναληπτική διαδικασία πεπερασμένων βημάτων, μέσω της οποίας συνδέονται τα σημεία που ελαχιστοποιούν το κόστος, είτε αυτό είναι η απόσταση είτε ο χρόνος εκτέλεσης. Η καλύτερη δυνατή σύνδεση των σημείων επιτυγχάνεται μέσω της επίλυσης του προβλήματος του περιοδεύοντος εμπόρου (Travelling Salesman Problem).

Μια αρκετά διαφορετική προσέγγιση παρουσιάζεται στο \cite{1262545}. Πιο συγκεκριμένα, χρησιμοποιούνται νευρωνικά δίκτυα για την πλήρη κάλυψη στατικού και μεταβαλλόμενου χώρου σε πραγματικό χρόνο. Η υλοποίηση αυτή εφαρμόζεται για χάρτη δύο διαστάσεων, μπορεί να καλύψει την ύπαρξη ενός ή περισσοτέρων ρομποτικών πρακτόρων και παρουσιάζει αξιόπιστα αποτελέσματα. Ο χώρος διακριτοποιείται με βάση το μέγεθος του ρομπότ και τα σημεία που προκύπτουν ενώνονται μέσω των νευρώνων του δικτύου. Η δυναμική μορφή του δικτύου μπορεί να μεταβάλλει στιγμιαία τα μονοπάτια που δημιουργούνται με βάση την προσθήκη ή την αφαίρεση εμποδίων.

Οι ευριστικές προσεγγίσεις βασίζονται στην κάλυψη του χώρου με διάφορα μοτίβα, όπως για παράδειγμα υλοποιείται στο \cite{DiFranco2016}. Στην περίπτωση αυτή, παρουσιάζεται ένας αλγόριθμος ο οποίος δημιουργεί ένα μονοπάτι που ικανοποιεί, εκτός των άλλων, και τους ενεργειακούς περιορισμούς του ρομπότ. Μετά την δημιουργία μιας πορείας που αποτελείται κυρίως από κινήσεις προς τα μπροστά και πίσω, εξετάζεται η ταχύτητα με την οποία πρέπει να κινείται το drone, ώστε να ελαχιστοποιείται η κατανάλωση ενέργειας.

