\chapter{Συμπεράσματα}
\label{chapter:conclusions}

Στο κεφάλαιο αυτό παρουσιάζονται τα συμπεράσματα που προέκυψαν από την έκβαση των πειραμάτων τόσο του συστήματος εντοπισμού θέσης, όσο και της κάλυψης χώρου. Για την αξιολόγηση των αποτελεσμάτων χρησιμοποιούμε τις τιμές του σφάλματος στην εκτίμηση θέσης και τον χρόνο και το ποσοστό κάλυψης για την πλήρη κάλυψη. Στη συνέχεια, αναφέρονται προβλήματα που παρουσιάστηκαν κατά τη διάρκεια των υλοποιήσεων και των πειραμάτων.

\section{Γενικά Συμπεράσματα}
Ο εντοπισμός της θέσης ενός drone σε κλειστό χώρο είναι ένα από τα βασικά προβλήματα που απασχολούν τις εφαρμογές των μη επανδρωμένων οχημάτων. Γνωρίζοντας την θέση του, το ρομπότ μπορεί να πραγματοποιήσει πληθώρα εφαρμογών που απαιτούν την ασφαλή και αξιόπιστη πλοήγηση στο χώρο. Στην παρούσα διπλωματική εργασία αναπτύχθηκε ένα σύστημα εντοπισμού θέσης που βασίζεται σε φίλτρο σωματιδίων και έχει ως είσοδο αποκλειστικά τις μετρήσεις του αισθητήρα απόστασης και ενός IMU. Επιπλέον, επιλύθηκε το πρόβλημα της πλήρους κάλυψης χώρου σε γνωστό τρισδιάστατο χάρτη μορφής OctoMap με την επιλογή των σημείων κάλυψης, αλλά και την ένωση τους με όσο το δυνατόν πιο βέλτιστο τρόπο.

Με βάση τα αποτελέσματα των πειραμάτων που αναλύθηκαν στο \autoref{chapter:experiments} και αφορούν το σφάλμα της εκτίμησης θέσης από την πραγματική, παρατηρήθηκαν τα εξής:
\begin{itemize}
    \item {Το σύστημα εντοπισμού θέσης εξαρτάται άμεσα από τον τρόπο κίνησης του drone, από την ταχύτητα του, καθώς και από την πορεία που ακολουθεί.}
    \item{Το μικρότερο σφάλμα στην εκτίμηση θέσης εμφανίζεται όταν το drone κινείται με χαμηλή ταχύτητα. Όσο χαμηλότερη, τόσο πιο σταθερή και ομαλή είναι η κίνησή του. Παρόλα αυτά, ακόμη και για υψηλότερες ταχύτητες το σφάλμα παραμένει σε σχετικά αποδεκτές τιμές.}
    \item{Στην περίπτωση όπου το drone διατηρεί σταθερό τον προσανατολισμό του, όπως συνέβαινε στην περίπτωση της κίνησης σε σπιράλ, επιτρέπεται η κίνηση σε υψηλότερες ταχύτητες, καθώς το σφάλμα παραμένει στα ίδια επίπεδα.}
    \item{Στην πιο ρεαλιστική περίπτωση, όπου το drone περιστρέφεται κατά την κίνηση του, η κίνηση πρέπει να είναι αργή και διατηρώντας όσο τον δυνατόν μεγαλύτερη απόσταση ασφαλείας από τα εμπόδια στο χώρο. Κάποιο στιγμιαίο σφάλμα στην εκτίμηση της θέσης μπορεί να οδηγήσει σε λανθασμένη πλοήγηση και συνεπώς σε σύγκρουση.}
    \item{Κατά την πλήρη κάλυψη χώρου, προτιμάται η χρήση ενός αισθητήρα με ευρύ οριζόντιο και κάθετο πεδίο όρασης, καθώς οδηγεί σε πιο γρήγορη και αποτελεσματικότερη κάλυψη του χώρου.}
    \item{Στην περίπτωση όπου ο χρόνος κάλυψης ενός χώρου έχει μεγαλύτερη σημασία είναι προτιμότερη η κάθετη ένωση των σημείων στο χώρο.}
    \item{Στην περίπτωση όπου το ποσοστό κάλυψης του χώρου έχει μεγαλύτερη σημασία, ανεξαρτήτως διάρκειας, η οριζόντια ένωση οδηγεί σε καλύτερα αποτελέσματα.}
\end{itemize}




% ----------------------------------------------------------------------------

\section{Προβλήματα}

Ένα βασικό πρόβλημα που παρουσιάστηκε κατά την υλοποίηση της διπλωματικής εργασίας ήταν η ορθή αναπαράσταση του χώρου σε μορφή OctoMap. Ο χάρτης λήφθηκε με χρήση του πακέτου SLAM RTAB-Map \cite{6459608} και ενός TurtleBot στο Gazebo\footnote{\href{https://wiki.ros.org/turtlebot\_gazebo}{https://wiki.ros.org/turtlebot\_gazebo}}. Η ορατότητα αυτού του επίγειου ρομπότ φτάνει μέχρι ένα συγκεκριμένο ύψος. Σημεία του χώρου τα οποία βρίσκονται σε μεγαλύτερο ύψος και είναι απαραίτητα για την πλοήγηση του drone, δεν είναι δυνατόν να συμπεριληφθούν στον χάρτη. Επίσης, η ύπαρξη θορύβου στην αναπαράσταση του χώρου αποτέλεσε πρόβλημα στην σωστή εκτίμηση του όγκου του χώρου, γεγονός που δυσκόλευε την διαδικασία αξιολόγησης της πλήρους κάλυψης. Μία λύση για το πρώτο πρόβλημα θα ήταν η χρήση ενός πακέτου που πραγματοποιεί SLAM με την χρήση ενός drone και θα παράγει εξίσου καλά αποτελέσματα. Για το δεύτερο, μια πρόταση είναι η χρήση μεθόδων για εξαγωγή χαρτών από τρισδιάστατα μοντέλα. 

Ένα ακόμη πρόβλημα είναι η ταχύτητα επεξεργασίας της πληροφορίας που δέχεται το drone. Παρόλο που η υλοποίηση δεν έχει υψηλές υπολογιστικές απαιτήσεις, η γρήγορη λήψη πληροφοριών οδηγεί στην αδυναμία χειρισμού της. Το γεγονός αυτό περιορίζει σημαντικά την ταχύτητα που μπορεί να κινηθεί το ρομπότ στο χώρο και να διατηρεί την ορθή εκτίμηση της θέσης του.
