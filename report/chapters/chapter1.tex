\chapter{Εισαγωγή}
\label{chapter:intro}

Η \emph{Ρομποτική} είναι η επιστήμη που ασχολείται με την αντίληψη και το χειρισμό του φυσικού κόσμου μέσω συσκευών που ελέγχονται από υπολογιστές. Τα παραδείγματα πετυχημένων ρομποτικών συστημάτων περιλαμβάνουν κινητές πλατφόρμες για πλανητική εξερεύνηση, βιομηχανικούς ρομποτικούς βραχίονες σε γραμμές συναρμολόγησης, αυτοκίνητα που κινούνται χωρίς οδηγό και βραχίονες που βοηθούν τους χειρουργούς. Τα ρομποτικά συστήματα βρίσκονται στον φυσικό κόσμο, αντιλαμβάνονται πληροφορίες για το περιβάλλον τους μέσω αισθητήρων και εκτελούν χειρισμούς στο περιβάλλον τους μέσω φυσικών δυνάμεων.

Στην επιστήμη της τεχνολογίας υπάρχει παρανόηση ως προς το τι είναι ρομπότ και τι δεν είναι. Ο ορισμός της λέξης ρομπότ εξελίχθηκε με την πάροδο του χρόνου, παράλληλα με τα άλματα της έρευνας και την πρόοδο της τεχνολογίας \cite{mataric2007}.
\begin{displayquote}
    Ρομπότ είναι ένα αυτόνομο σύστημα, το οποίο υπάρχει στο φυσικό κόσμο, αισθάνεται το περιβάλλον του και μπορεί να δράσει σε αυτό ώστε να πετύχει κάποιους στόχους. 
\end{displayquote}

Ένα \emph{αυτόνομο} ρομπότ δρα βάσει των δικών του αποφάσεων και δεν ελέγχεται από κάποιον άνθρωπο. Η ικανότητα αίσθησης του περιβάλλοντος υποδηλώνει την ύπαρξη \emph{αισθητήρων}, δηλαδή μέσων αντίληψης (π.χ., ακοή, αφή, όραση, όσφρηση, κ.λπ) που τροφοδοτούν το ρομπότ με πληροφορίες από τον κόσμο. Μια μηχανή που δεν δρα (δηλαδή δεν κινείται και δεν επηρεάζει τον κόσμο πράττωντας/αλλάζοντας κάτι) δεν θεωρείται ρομπότ. \\

Το 1950 ο Isaac Asimov διατύπωσε τους τρεις νόμους της Ρομποτικής, οι οποίοι βρίσκονται υπό συνεχή αναθεώρηση μέχρι σήμερα \cite{mccauley2007} και είναι οι εξής:
\begin{displayquote}
 \begin{itemize}
  \item{Το ρομπότ δε θα κάνει κακό σε άνθρωπο, ούτε με την αδράνειά του θα επιτρέψει να βλαφτεί ανθρώπινο ον}
  \item{Το ρομπότ πρέπει να υπακούει τις διαταγές που του δίνουν οι άνθρωποι, εκτός αν αυτές οι διαταγές έρχονται σε αντίθεση με τον πρώτο νόμο}
  \item{Το ρομπότ οφείλει να προστατεύει την ύπαρξή του, εφόσον αυτό δεν συγκρούεται με τον πρώτο και τον δεύτερο νόμο}
\end{itemize}

\end{displayquote}


\section{Περιγραφή του Προβλήματος}
\label{section:problem_description}

Ως γνωστόν, η αγορά και η χρήση των \emph{Μη Επανδρωμένων Αεροσκαφών}, γνωστά και ως drones, έχει αυξηθεί κατακόρυφα τα τελευταία χρόνια. Αρχικά, τα drones είχαν κυρίως στρατιωτική χρήση, κάτι που δημιουργεί αρκετές ηθικές και νομικές προκλήσεις, αλλά και προβληματισμό σχετικά με την παραβίαση της ιδιωτικότητας στην σύγχρονη εποχή. Επίσης, ενώ στην αρχή ήταν απαραίτητη η ύπαρξη κάποιου καταρτισμένου χειριστή, πλέον τα περισσότερα είναι αυτόνομα, μπορούν να ακολουθούν κάποιο προκαθορισμένο πλάνο πτήσης και να προσαρμόζονται στις μεταβολές του περιβάλλοντος. Η επιστημονική κοινότητα θεωρεί τα drones μία από τις τεχνολογίες του μέλλοντος, καθώς οι δυνατότητές τους επιτρέπουν την ανάπτυξη ρομποτικών εφαρμογών που επιλύουν προβλήματα σε διάφορους κλάδους, από την εξερεύνηση δυσπρόσιτων περιοχών μέχρι και την κινηματογράφηση αθλητικών εκδηλώσεων.

Κάποιες χαρακτηριστικές, αλλά όχι οι μόνες, περιπτώσεις, όπου τα μη επανδρωμένα αεροσκάφη μπορούν να χρησιμοποιηθούν είναι οι εξής:

\begin{itemize}

  \item{Η επίβλεψη και συντήρηση μεγάλων εκτάσεων ή κτισμάτων, π.χ. γέφυρες, τούνελ, υπόγεια ορυχεία κ.α.}
  \item{Η παροχή ανθρωπιστικής βοήθειας σε δυσπρόσιτες περιοχές μετά από φυσικές καταστροφές}
  \item{Η συνεχής απογραφή προϊόντων σε μεγάλες αποθήκες χωρίς την ανθρώπινη παρέμβαση} 
   \item{Ο κλάδος της γεωργίας και η παρακολούθηση αγροτικών εκτάσεων}
  \item{Η μεταφορά φορτίων και δεμάτων μικρού όγκου}

\end{itemize}

Το πρόβλημα στο οποίο αναφέρεται η συγκεκριμένη εργασία είναι αυτό της αυτόνομης συνεχούς απογραφής προϊόντων σε οποιονδήποτε χώρο. Η απογραφή από ανθρώπους είναι μία επίπονη διαδικασία, η οποία σχετίζεται με επιβάρυνση της υγείας τους, λόγω της επαναλαμβανόμενης κίνησης που καλούνται να κάνουν και ενδεχόμενα σφάλματα. Με την χρήση των drones η διαδικασία αυτή απλουστεύεται με επιθυμητό αποτέλεσμα τον προσδιορισμό της θέσης όλων των προϊόντων με ακρίβεια μερικών εκατοστών. Το πρόβλημα αυτό αποτελείται από τα εξής υπο-προβλήματα: τον εντοπισμό θέσης του drone στον κλειστό χώρο, τον εντοπισμό θέσης των ετικετών των προϊόντων και της πλήρης κάλυψης του χώρου αυτόνομα.

Με τον όρο εντοπισμό θέσης ενός ρομποτικού οχήματος, αναφερόμαστε στον υπολογισμό της θέσης και του προσανατολισμού του ρομπότ ως προς έναν δεδομένο χάρτη. Η χρήση των μη επανδρωμένων αεροσκαφών σε εσωτερικούς χώρους παρουσιάζει μεγάλο ενδιαφέρον, καθώς δεν είναι δυνατή η χρήση του αισθητήρα GPS και ως συνέπεια ο άμεσος εντοπισμός της θέσης του drone στο χώρο. Οποιαδήποτε διαδικασία που θα επιτελεί το drone σε έναν κλειστό χώρο, απαιτεί πολύ καλή αντίληψη του περιβάλλοντος που βρίσκεται, άμεση απόκριση σε μεταβολές του, σταθερή και ασφαλή πλοήγηση. Επίσης, είναι γνωστό ότι όλοι οι αισθητήρες περιέχουν θόρυβο, συνεπώς δεν προσφέρουν απόλυτα αξιόπιστα αποτελέσματα από μόνοι τους. Για τους παραπάνω λόγους, απαιτείται η πλήρης εκμετάλλευση των υπόλοιπων διαθέσιμων αισθητήρων. 

Στην περίπτωση μας, οι αισθητήρες αποτελούνται από μία κάμερα, από έναν αισθητήρα απόστασης (laser) ο οποίος μας δίνει ορισμένες αποστάσεις περιμετρικά του drone, έναν ακόμη αισθητήρα laser που υπολογίζει το ύψος που βρίσκεται και έναν αισθητήρα αδρανειακής μέτρησης (Inertial Measurement Unit). Το περιβάλλον στο οποίο βρίσκεται το drone θεωρείται γνωστό και ο χάρτης αυτού είναι διαθέσιμος εκ των προτέρων σε μορφή OctoMap \cite{hornung13auro}.

Εκτός από την ορθή εκτίμηση της θέσης του, το μη επανδρωμένο αεροσκάφος πρέπει να ακολουθεί ένα προκαθορισμένο πλάνο πτήσης, το οποίο διαμορφώνεται σύμφωνα με συγκεκριμένα σημεία του χάρτη που παρουσιάζουν κάποιο ενδιαφέρον. Με τον τρόπο αυτό, εξασφαλίζεται η πλήρης τρισδιάστατη κάλυψη του χώρου. Στη συγκεκριμένη περίπτωση μελέτης, το πλάνο πτήσης μπορεί να διαμορφωθεί βάση των προϊόντων που υφίστανται προς απογραφή στο χώρο της αποθήκης. Οι αλγόριθμοι πλήρους κάλυψης ενός χώρου, προϋποθέτουν την ύπαρξη ενός μονοπατιού, το οποίο είναι προσπελάσιμο σε πεπερασμένο χρόνο. Η αξιολόγηση της κάλυψης του χώρου γίνεται με την χρήση μίας RFID κεραίας και αναγνώστη, μέσω του οποίου υπολογίζεται η θέση των αντικειμένων που βρίσκονται στο χώρο και στη συνέχεια υπολογίζεται το ποσοστό των αντικειμένων που εντοπίστηκαν από τον αναγνώστη ως προς τον συνολικό αριθμό αντικειμένων.  


\section{Σκοπός - Συνεισφορά της Διπλωματικής Εργασίας}
\label{section:contribution}

Σκοπός της παρούσας διπλωματικής εργασίας είναι να παρουσιάσει ένα ολοκληρωμένο σύστημα εντοπισμού θέσης και πλήρους κάλυψης ενός γνωστού χώρου χρησιμοποιώντας ένα μη επανδρωμένο αεροσκάφος. Το σύστημα αυτό θα πρέπει να παρέχει στους αλγορίθμους πλοήγησης μια αξιόπιστη εκτίμηση της θέσης του quadcopter, καθώς και τα σημεία τα οποία θα πρέπει να διασχίσει, από τα οποία αποτελείται το πλάνο πτήσης του.

Οι μετρήσεις οι οποίες προέρχονται από τον αισθητήρα απόστασης και το IMU συνδυάζονται, έτσι ώστε να εκτιμηθεί η θέση του αεροσκάφους στο χώρο. Η πληροφορία αυτή, παρέχεται στον αλγόριθμο εντοπισμού θέσης, που βασίζεται σε ένα φίλτρο σωματιδίων.

Για την δημιουργία της πορείας που πρέπει να ακολουθήσει το drone ώστε να φτάσει σε κάποιο στόχο, χρησιμοποιείται ο αλγόριθμος RRT*, χρησιμοποιώντας τον χάρτη σε μορφή OctoMap για την αποφυγή των εμποδίων. Το μονοπάτι που προκύπτει από αυτόν, αν χρειαστεί, ομαλοποιείται χρησιμοποιώντας συναρτήσεις B-spline. Στη συνέχεια, με τη χρήση ενός PID ελεγκτή θέσης το drone κινείται προς το στόχο αυτό.

Τέλος, για την πλήρη κάλυψη του χώρου, εξετάζεται ο χάρτης, ώστε να βρεθούν τα σημεία τα οποία προσφέρουν την καλύτερη δυνατή θέαση προς τα αντικείμενα. Έχοντας τα σημεία αυτά, χρησιμοποιείται ο αλγόριθμος πλησιέστερου γείτονα (Nearest Neighbor) σε συνδυασμό με τον αλγόριθμο αναρρίχησης λόφων (Hill-Climbing Search) για να προκύψει η τελική σειρά αυτών που δίνει το συνολικό μονοπάτι.

Ο κώδικας που αναπτύχθηκε για την επίλυση του προβλήματος εφαρμόστηκε σε περιβάλλον προσομοίωσης και για αισθητήρες με συγκεκριμένα χαρακτηριστικά. Παρ' όλα αυτά, η μεταφορά σε πραγματικό drone δεν απαιτεί ιδιαίτερη προσαρμογή, καθώς τα κύρια χαρακτηριστικά έχουν παραμετροποιηθεί για εύκολη μεταβολή του συστήματος.
\section{Διάρθρωση της Αναφοράς}
\label{section:layout}

Η διάρθρωση της παρούσας διπλωματικής εργασίας είναι η εξής:

\begin{itemize}
  \item{\textbf{Κεφάλαιο \ref{chapter:sota}:}
      Γίνεται ανασκόπησή της ερευνητικής περιοχής που αφορά την εύρεση θέσης ενός μη επανδρωμένου αεροσκάφους σε γνωστό χώρο και την πλήρη κάλυψη του χώρου από αυτό.
    }
  \item{\textbf{Κεφάλαιο \ref{chapter:theory_tools}:} Περιγράφονται τα βασικά θεωρητικά στοιχεία στα οποία βασίστηκαν οι υλοποιήσεις, καθώς και τα εργαλεία που χρησιμοποιήθηκαν. Πιο συγκεκριμένα, περιγράφεται η αρχή λειτουργίας των μη επανδρωμένων αεροσκαφών και ο τρόπος λειτουργίας ενός PID ελεγκτή. Επίσης, περιγράφεται το μεσολειτουργικό σύστημα ROS, πάνω στο οποίο βασίστηκε όλη η υλοποίηση, το σύνολο ROS πακέτων Hector Quadrotor καθώς και οι βιβλιοθήκες OctoMap, Particle Filter και Open Motion Planning Library.
    }
  \item{\textbf{Κεφάλαιο \ref{chapter:implementations}:} Πλήρης περιγραφή των υλοποιήσεων των αλγορίθμων εύρεσης θέσης, πλοήγησης και πλήρους κάλυψης χώρου. 
    }
  \item{\textbf{Κεφάλαιο \ref{chapter:experiments}:} Παρουσιάζεται αναλυτικά η μεθοδολογία των
      πειραμάτων και τα αποτελέσματα.
    }
  \item{\textbf{Κεφάλαιο \ref{chapter:conclusions}:} Παρουσιάζονται τα τελικά συμπεράσματα.
    }
  \item{\textbf{Κεφάλαιο \ref{chapter:future_work}:} Αναφέρονται τα
      προβλήματα που προέκυψαν και προτείνονται θέματα για μελλοντική
      μελέτη, αλλαγές και επεκτάσεις.
    }
\end{itemize}
