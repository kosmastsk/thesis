\chapter{Μελλοντικές επεκτάσεις}
\label{chapter:future_work}

Μια σημαντική βελτίωση της παρούσας εργασίας θα ήταν η υλοποίηση του Adaptive Particle Filter, μιας μεθόδου η οποία προσαρμόζει τον αριθμό των σωματιδίων που χρησιμοποιούνται ανάλογα με το βάρος αυτών. Με τον τρόπο αυτό μπορεί να μειωθεί η υπολογιστική ισχύς που απαιτείται, καθώς όταν υπάρχει μεγάλη βεβαιότητα για την εκτίμηση της θέσης θα υπάρχουν λιγότερα σωματίδια για επεξεργασία. Επιπλέον, θα μπορούσε να αντιμετωπιστεί και η περίπτωση του καθολικού εντοπισμού θέσης, με την αρχικοποίηση περισσότερων σωματιδίων σε ολόκληρο το χώρο και τη μείωση τους αργότερα. Επίσης, για την χρήση του συστήματος εντοπισμού θέσης εκτός περιβάλλοντος προσομοίωσης είναι σημαντικό να χρησιμοποιηθεί οπτική οδομετρία, μέσω της κάμερας που ήδη διαθέτει το ρομπότ. Παρόλο που θα αυξηθούν οι υπολογιστικές απαιτήσεις, πιθανόν θα επιφέρει καλύτερα αποτελέσματα. Επίσης, μπορεί να μελετηθεί η χρήση της κάμερας που δεν διαθέτει καμία πληροφορία βάθους, για την υποστήριξη του συστήματος εντοπισμού θέσης. Η αντιστοίχιση των λήψεων με κάποιον τρόπο με τμήματα του χάρτη, μπορεί να προσδόσει σημαντική πληροφορία.

Όσον αφορά το σύστημα πλήρους κάλυψης χώρου, θα μπορούσε να επεκταθεί με την υλοποίηση μιας μεθόδου που θα αποφασίζει σε πραγματικό χρόνο τα σημεία τα οποία θα επισκεφτεί. Με βάση τις περιοχές που έχουν ήδη καλυφθεί και με τους περιορισμούς χρόνου που πιθανόν υπάρχουν, μπορεί να επιλέγει τα σημεία τα οποία θα επιφέρουν καλύτερο συνολικό αποτέλεσμα και όχι απλά να ακολουθεί το αρχικό πλάνο. Επίσης, παρουσιάζει μεγάλο ενδιαφέρον η μελέτη της ύπαρξης μοτίβων ανάλογα με την κάθε περίπτωση και η σημασιολογικός διαχωρισμός του χώρου.

Μια επίσης ενδιαφέρουσα υλοποίηση, θα ήταν η κάλυψη του χώρου από πολλαπλά drone. Ο συγχρονισμός αυτών στο χώρο και η εύρεση ενός τρόπου κάλυψης, ο οποίος θα χρησιμοποιούσε μη επικαλυπτόμενα μονοπάτια μπορεί να οδηγήσει στην πλήρη κάλυψη του χώρου σε μειωμένο χρόνο και πιθανόν με μεγαλύτερη ακρίβεια. 

Επιπλέον, είναι σημαντικό να μελετηθούν οι ενεργειακές ανάγκες του drone. Στην συγκεκριμένη εργασία, οι υλοποίησεις δεν λαμβάνουν υπόψη τους τη διάρκεια πτήσης του drone, ούτε την επιρροή της ταχύτητας και του φορτίου στην κατανάλωση ενέργειας. Σε μία ρεαλιστική κατάσταση όμως, ο χρόνος πτήσης είναι περιορισμένος και θα πρέπει να συμπεριλαμβάνεται στον σχεδιασμό μονοπατιού κάλυψης του χώρου. 