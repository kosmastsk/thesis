\section{The Open Motion Planning Library (OMPL)}
\label{section:ompl}

Η βιβλιοθήκη OMPL\footnote{\href{https://ompl.kavrakilab.org/}{https://ompl.kavrakilab.org/}} περιέχει πληθώρα αλγορίθμων που σχετίζονται με το πρόβλημα της πλοήγησης στις ρομποτικές εφαρμογές \cite{6377468}. Οι αλγόριθμοι που περιέχει όπως τα Rapidly-expanding Random Trees, Probabilistic Roadmap Method και άλλοι, βασίζονται σε δειγματοληπτικές μεθόδους και περιέχουν επιλύσεις για χώρο κάθε διάστασης. Είναι ανοικτού κώδικα, γραμμένη σε C++ και πλήρως συμβατή με το ROS, ενώ έχει χρησιμοποιηθεί από την επιστημονική κοινότητα τόσο για εκπαιδευτικούς όσο και για ερευνητικούς σκοπούς.

\begin{figure}[!ht]
    \centering
    \includegraphics[width=0.8\textwidth]{./images/chapter3/gui_path.png}
    \caption{Σχεδιασμός μονοπατιού με την OMPL} 
    \label{fig:ompl}
\end{figure}

Η βασική ιδέα στην οποία βασίζονται οι αλγόριθμοι αυτοί είναι ο διαχωρισμός του χώρου σε σημεία καταστάσεων και η ένωση των σημείων αυτών ως κορυφές σε έναν γράφο. Οι ακμές του γράφου υποδηλώνουν όλα τα εφικτά μονοπάτια μέσα στον χώρο. Η μορφή των σημείων εξαρτάται από το αντικείμενο για το οποίο εφαρμόζεται το path planning και τις δυνατές κινήσεις του, δηλαδή για ένα επίγειο ρομπότ αποτελείται από την κίνηση ως προς τους άξονες $x$ και $y$, ενώ για ένα drone θα είχε επιπλέον την κίνηση στον άξονα $z$, δηλαδή το ύψος στο οποίο βρίσκεται. Επίσης, περιέχει έλεγχο των καταστάσεων, ώστε να αποκλειστούν θέσεις οι οποίες είναι πιθανό να επιφέρουν σύγκρουση ή βρίσκονται εκτός των ορίων του περιβάλλοντος.

Η συγκεκριμένη βιβλιοθήκη προτιμήθηκε καθώς μπορεί να δεχτεί ως είσοδο έναν χάρτη σε μορφή OctoMap και να πραγματοποιήσει σχεδιασμό διαδρομής μέσα σε αυτόν. Επίσης, μετά τον σχεδιασμό των σημείων που απαιτούνται για την επίτευξη του στόχου, πραγματοποιείται μια εξομάλυνση του μονοπατιού, ανάλογα με την μετρική που δείχνει το πόσο ομαλό είναι το μονοπάτι. Με τον τρόπο αυτό, εξασφαλίζουμε την πλοήγηση με αποφυγή των γνωστών στο χώρο εμποδίων. Στο \autoref{fig:ompl} βλέπουμε ενδεικτικά την χρήση της βιβλιοθήκης για την δημιουργία ενός μονοπατιού στο χώρο.

Ο αλγόριθμος που χρησιμοποιήθηκε για την δημιουργία των μονοπατιών είναι ο RRT* (Optimal RRT) \cite{DBLP:journals/corr/abs-1105-1186}. Πρόκειται για μια βελτιωμένη εκδοχή του αλγορίθμου RRT \cite{lavalle1999}, ο οποίος εγγυάται την σύγκλιση σε βέλτιστη λύση, ενώ ο χρόνος εκτέλεσης του είναι σταθερός παράγοντας του χρόνου εκτέλεσης του RRT. 