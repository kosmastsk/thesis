\section{Robot Operating System (ROS)}
\label{section:ROS}

Το Robot Operating System \cite{ros2009} είναι το πιο διαδεδομένο μεσολειτουργικό σύστημα για ρομποτικές εφαρμογές. Αποτελείται από ένα σύνολο βιβλιοθηκών και εργαλείων τα οποία στοχεύουν στην απλοποίηση της δημιουργίας πολύπλοκων και αξιόπιστων εφαρμογών που περιλαμβάνουν τη χρήση ρομποτικών πρακτόρων. Το ROS είναι ανοικτού κώδικα και έχει διαμορφωθεί με την συνεισφορά πολλών εθελοντών ανά τον κόσμο. Ο λόγος που δημιουργήθηκε είναι το γεγονός ότι η δημιουργία λογισμικού για ρομποτ είναι αρκετά δύσκολη, ιδιαίτερα όσο αυξάνεται το εύρος των ρομποτικών εφαρμογών. Διαφορετικά ρομπότ διαθέτουν και διαφορετικό υλικό (hardware) και έχουν διαφορετικές λειτουργικές απαιτήσεις ανάλογα με το σκοπό χρήσης τους. Συνεπώς, η δημιουργία ενός κοινού πλαισίου ανάπτυξης λογισμικού για ρομπότ, οδήγησε στην απλοποίηση αυτής της διαδικασίας με την ύπαρξη έτοιμων τμημάτων κώδικα που εκτελούν βασικές λειτουργίες σε κάθε εφαρμογή και την επαναχρησιμοποίησή τους.

Το ROS παρέχει όσα θα περίμενε κανείς και από ένα λειτουργικό σύστημα, δηλαδή αφαίρεση υλικού (hardware), έλεγχο συσκευών σε χαμηλό επίπεδο, ανταλλαγή μηνυμάτων μεταξύ των διεργασιών και διαχείριση πακέτων. Η δομή του βασίζεται στην αρχιτεκτονική peer-to-peer, όπου η κάθε διαδικασία είναι ένας κόμβος (node) \footnote{\href{http://wiki.ros.org/Nodes}{http://wiki.ros.org/Nodes}} στον γράφο. Οι κόμβοι μπορούν να είναι κατανεμημένοι σε διαφορετικά συστήματα, έχουν όμως τη δυνατότητα να επικοινωνούν μέσω της δομής επικοινωνίας του ROS. Υπάρχουν υλοποιημένες διάφορες μορφές επικοινωνίας, όπως η σύγχρονη επικοινωνία μέσω των services\footnote{\href{http://wiki.ros.org/Services}{http://wiki.ros.org/Services}}, η ασύγχρονη ροή δεδομένων σε topic\footnote{\href{http://wiki.ros.org/Topics}{http://wiki.ros.org/Topics}} και η αποθήκευση δεδομένων στον Parameter Server\footnote{\href{http://wiki.ros.org/Parameter\%20Server}{http://wiki.ros.org/Parameter\%20Server}}. Παρόλο που το ROS δεν είναι ένα σύστημα πραγματικού χρόνου, μπορεί να χρησιμοποιηθεί με κώδικα πραγματικού χρόνου, γεγόνός που το καθιστά αξιόπιστο για λειτουργία σε ρομποτικά συστήματα που απαιτούν άμεση απόκριση.

Ενδεικτικά, η αρχιτεκτονική του ROS φαίνεται στο \autoref{fig:ros_arch}. Ο κόμβος ROS Master δημιουργεί και συντηρεί τον αρχιτεκτονικό γράφο του συστήματος. Επιτρέπει σε κάθε ξεχωριστό ROS κόμβο να εντοπίσει κάποιον άλλον με τον οποίο θέλει να έρθει σε επικοινωνία.

\begin{figure}[!ht]
    \centering
    \includegraphics[width=1.0\textwidth]{./images/chapter3/ros101-2.png}
    \caption{Η αρχιτεκτονική του ROS} 
    Πηγή: \href{https://robohub.org/ros-101-intro-to-the-robot-operating-system/}{https://robohub.org/ros-101-intro-to-the-robot-operating-system/}
    \label{fig:ros_arch}
\end{figure}

% maybe add topics and messages we used and TF ?? 