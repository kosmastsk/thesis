\section{Αρχή λειτουργίας των UAV}
\label{section:quadcopter}

Τα μη επανδρωμένα οχήματα χωρίζονται σε διαφορετικές κατηγορίες, ανάλογα με τον αριθμό και τη διάταξη των κινητήρων τους. Στην συγκεκριμένη εργασία, θα ασχοληθούμε με το τετρακόπτερο (Quadcopter), του οποίου η μορφή φαίνεται στο παρακάτω σχήμα.
\begin{figure}[!ht]
    \centering
    \includegraphics[scale=0.5]{./images/chapter3/QuadRotorPlus.png}
    \caption{Δομή ενός Quadcopter} 
    Πηγή: \href{https://dev.px4.io/en/airframes/airframe\_reference.html}{https://dev.px4.io/en/airframes/airframe\_reference.html}
    \label{fig:drone_1}
\end{figure}

Όπως βλέπουμε, διαθέτει τέσσερις έλικες τοποθετημένους σε ίσες αποστάσεις και συμμετρικά κατανεμημένους από το κέντρο μάζας του drone, οι οποίοι είναι υπεύθυνοι για όλες τις λειτουργίες, όπως την κάθετη απογείωση και την προσγείωση του οχήματος. Διαθέτει έξι βαθμούς ελευθερίας και τέσσερις ελεγχόμενες μεταβλητές, όσες και ο αριθμός των κινητήρων. Μέσω αυτών επιτυγχάνεται ο έλεγχος της θέσης και του ύψους στο οποίο βρίσκεται το Quadcopter. 

Οι πιο χαρακτηριστικές λειτουργίες είναι η ανύψωση από το έδαφος, η προσγείωση και η σταθερή αιώρηση (hover). Για την ανύψωση από το έδαφος απαιτείται οι έλικες να κινούνται με την ίδια ταχύτητα και με την κατεύθυνση που φαίνεται στο \autoref{fig:drone_1}, ενώ η προσγείωση επιτυγχάνεται με την μείωση της ταχύτητας των στροφέων. Η λειτουργία hover, δηλαδή η σταθεροποίηση σε κάποιο συγκεκριμένο ύψος, χρειάζεται την κίνηση των στροφέων με τέτοια ταχύτητα, ώστε η βαρυτική δύναμη να είναι ίση με τη δύναμη που ωθεί το drone προς τα πάνω. Όσον αφορά την περιστροφή του drone, οι έλικες πρέπει να κινούνται με αντίθετη φορά και διαφορετικές ταχύτητες ανά δύο. Συγκεκριμένα, στο \autoref{fig:drone_moves} για μία αριστερόστροφη περιστροφή οι έλικες 2 και 4 θα περιστραφούν πιο γρήγορα από τους έλικες 1 και 3. Για την κίνηση προς τα δεξιά, αυξάνουμε την περιστροφική ταχύτητα στους έλικες που βρίσκονται στην πλευρά προς την οποία θέλουμε να κινηθούμε.

\begin{figure}[!ht]
    \centering
    \includegraphics[width=0.9\textwidth]{./images/chapter3/drone_movements.png}
    \caption{Περιστροφή και ευθεία κίνηση ενός Quadcopter} 
    \label{fig:drone_moves}
\end{figure}

Οι κινήσεις οι οποίες μπορεί να πραγματοποιήσει το Quadcopter αναλύονται ως προς τους τρεις άξονες x,y και z με τα roll, pitch και yaw αντίστοιχα. Pitch είναι η κίνηση με την οποία το drone κινείται είτε μπροστά είτε πίσω, roll είναι η κίνηση δεξιά ή αριστερά, ενώ τέλος το yaw ορίζει την κατεύθυνση του drone.