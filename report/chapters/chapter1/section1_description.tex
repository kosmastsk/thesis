\section{Περιγραφή του Προβλήματος}
\label{section:problem_description}

Ως γνωστόν, η αγορά και η χρήση των \emph{Μη Επανδρωμένων Αεροσκαφών}, γνωστά και ως drones, έχει αυξηθεί κατακόρυφα τα τελευταία χρόνια. Αρχικά, τα drones είχαν κυρίως στρατιωτική χρήση, κάτι που δημιουργεί αρκετές ηθικές και νομικές προκλήσεις, αλλά και προβληματισμό σχετικά με την παραβίαση της ιδιωτικότητας στην σύγχρονη εποχή. Επίσης, ενώ στην αρχή ήταν απαραίτητη η ύπαρξη κάποιου καταρτισμένου χειριστή, πλέον τα περισσότερα είναι αυτόνομα, μπορούν να ακολουθούν κάποιο προκαθορισμένο πλάνο πτήσης και να προσαρμόζονται στις μεταβολές του περιβάλλοντος. Η επιστημονική κοινότητα θεωρεί τα drones μία από τις τεχνολογίες του μέλλοντος, καθώς οι δυνατότητές τους επιτρέπουν την ανάπτυξη ρομποτικών εφαρμογών που επιλύουν προβλήματα σε διάφορους κλάδους, από την εξερεύνηση δυσπρόσιτων περιοχών μέχρι και την κινηματογράφηση αθλητικών εκδηλώσεων.

Κάποιες χαρακτηριστικές, αλλά όχι οι μόνες, περιπτώσεις, όπου τα μη επανδρωμένα αεροσκάφη μπορούν να χρησιμοποιηθούν είναι οι εξής:

\begin{itemize}

  \item{Η επίβλεψη και συντήρηση μεγάλων εκτάσεων ή κτισμάτων, π.χ. γέφυρες, τούνελ, υπόγεια ορυχεία κ.α.}
  \item{Η παροχή ανθρωπιστικής βοήθειας σε δυσπρόσιτες περιοχές μετά από φυσικές καταστροφές}
  \item{Η συνεχής απογραφή προϊόντων σε μεγάλες αποθήκες χωρίς την ανθρώπινη παρέμβαση} 
   \item{Ο κλάδος της γεωργίας και η παρακολούθηση αγροτικών εκτάσεων}
  \item{Η μεταφορά φορτίων και δεμάτων μικρού όγκου}

\end{itemize}

Το πρόβλημα στο οποίο αναφέρεται η συγκεκριμένη εργασία είναι αυτό της αυτόνομης συνεχούς απογραφής προϊόντων σε οποιονδήποτε χώρο. Η απογραφή από ανθρώπους είναι μία επίπονη διαδικασία, η οποία σχετίζεται με επιβάρυνση της υγείας τους, λόγω της επαναλαμβανόμενης κίνησης που καλούνται να κάνουν και ενδεχόμενα σφάλματα. Με την χρήση των drones η διαδικασία αυτή απλουστεύεται με επιθυμητό αποτέλεσμα τον προσδιορισμό της θέσης όλων των προϊόντων με ακρίβεια μερικών εκατοστών. Το πρόβλημα αυτό αποτελείται από τα εξής υπο-προβλήματα: τον εντοπισμό θέσης του drone στον κλειστό χώρο, τον εντοπισμό θέσης των ετικετών των προϊόντων και της πλήρης κάλυψης του χώρου αυτόνομα.

Με τον όρο εντοπισμό θέσης ενός ρομποτικού οχήματος, αναφερόμαστε στον υπολογισμό της θέσης και του προσανατολισμού του ρομπότ ως προς έναν δεδομένο χάρτη. Η χρήση των μη επανδρωμένων αεροσκαφών σε εσωτερικούς χώρους παρουσιάζει μεγάλο ενδιαφέρον, καθώς δεν είναι δυνατή η χρήση του αισθητήρα GPS και ως συνέπεια ο άμεσος εντοπισμός της θέσης του drone στο χώρο. Οποιαδήποτε διαδικασία που θα επιτελεί το drone σε έναν κλειστό χώρο, απαιτεί πολύ καλή αντίληψη του περιβάλλοντος που βρίσκεται, άμεση απόκριση σε μεταβολές του, σταθερή και ασφαλή πλοήγηση. Επίσης, είναι γνωστό ότι όλοι οι αισθητήρες περιέχουν θόρυβο, συνεπώς δεν προσφέρουν απόλυτα αξιόπιστα αποτελέσματα από μόνοι τους. Για τους παραπάνω λόγους, απαιτείται η πλήρης εκμετάλλευση των υπόλοιπων διαθέσιμων αισθητήρων. 

Στην περίπτωση μας, οι αισθητήρες αποτελούνται από μία κάμερα, από έναν αισθητήρα απόστασης (laser) ο οποίος μας δίνει ορισμένες αποστάσεις περιμετρικά του drone, έναν ακόμη αισθητήρα laser που υπολογίζει το ύψος που βρίσκεται και έναν αισθητήρα αδρανειακής μέτρησης (Inertial Measurement Unit). Το περιβάλλον στο οποίο βρίσκεται το drone θεωρείται γνωστό και ο χάρτης αυτού είναι διαθέσιμος εκ των προτέρων σε μορφή OctoMap \cite{hornung13auro}.

Εκτός από την ορθή εκτίμηση της θέσης του, το μη επανδρωμένο αεροσκάφος πρέπει να ακολουθεί ένα προκαθορισμένο πλάνο πτήσης, το οποίο διαμορφώνεται σύμφωνα με συγκεκριμένα σημεία του χάρτη που παρουσιάζουν κάποιο ενδιαφέρον. Με τον τρόπο αυτό, εξασφαλίζεται η πλήρης τρισδιάστατη κάλυψη του χώρου. Στη συγκεκριμένη περίπτωση μελέτης, το πλάνο πτήσης μπορεί να διαμορφωθεί βάση των προϊόντων που υφίστανται προς απογραφή στο χώρο της αποθήκης. Οι αλγόριθμοι πλήρους κάλυψης ενός χώρου, προϋποθέτουν την ύπαρξη ενός μονοπατιού, το οποίο είναι προσπελάσιμο σε πεπερασμένο χρόνο. Η αξιολόγηση της κάλυψης του χώρου γίνεται με την χρήση μίας RFID κεραίας και αναγνώστη, μέσω του οποίου υπολογίζεται η θέση των αντικειμένων που βρίσκονται στο χώρο και στη συνέχεια υπολογίζεται το ποσοστό των αντικειμένων που εντοπίστηκαν από τον αναγνώστη ως προς τον συνολικό αριθμό αντικειμένων.  

